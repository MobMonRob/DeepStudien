\chapter{Einleitung}

\section{Motivation}

Mit kontinuierlicher Weiterentwicklung der Oberflächen-Technologie SAP-UI5 und fehlender Abwärtskompatibilität steigt die Notwendigkeit von qualitätssichernden Maßnahmen um den laufenden Betrieb von Fiori Apps zu gewährleisten.
Dabei muss an zwei Punkten angesetzt werden. Zum einen an der Qualität der bereits entwickelten und produktiven Apps und zum anderen an der Neuentwicklungen.
Für ersteres existieren bereits Ansätze, es fehlt allerdings eine Lösung, die ein übergreifen-des Monitoring bietet und so möglichst jeden Aspekt der Entwicklung betrachtet anstatt sich auf einzelne zu konzentrieren.

\section{Ziel der Arbeit}

Ziel dieser Arbeit ist zunächst eine Lösung zu entwickeln, die es ermöglicht den App-Bestand eines Gateway-Systems hinsichtlich dessen Qualität zu untersuchen. Hierfür müssen Qualitätsmerkmale anhand der Erfahrung von Mitarbeitern des MaxAttention Supports und den SAP Guidelines definiert werden. 
Nach der Definition der Qualitätsmerkmale muss ein Lösungskonzept zur schnellen und übersichtlichen Auswertung der Merkmale entwickelt werden. Dazu müssen Entscheidungen hinsichtlich der zu wählenden Technologien getroffen und der Funktionsumfang festgelegt werden.

\section{Aufbau}
Die Einleitung umreißt als erstes das generelle Ziel dieser Arbeit, bevor im anschließenden Kapitel die für diese Arbeit relevanten Grundlagen gelegt werden. Diese beinhalten zum einen die, von SAP empfohlenen, Guidelines zur Architektur von FIORI Apps, zum anderen die technischen Besonderheiten, die der Implementierung der neuen Monitoring-Funktionalität zu Grunde liegen.
Anschließend und aufbauend auf die Grundlagen folgt in Kapitel 3 die Analyse der bereits angesprochenen Probleme, die die noch einmal die Wichtigkeit einer Monitoring-Lösung verdeutlichen. Abschließend werden die Ergebnisse der Analyse in einem Fazit als Anforderungen an die Monitoring-Lösung definiert.
Im Konzept-Kapitel werden die definierten Anforderungen aufgegriffen und die Umsetzung konzeptionell geplant. Diese Planung fasst die drei Teile Entscheidung für eine Technologie, Möglichkeiten zur Informationsbeschaffung und Darstellung unter sich zusammen. Nach der konzeptuellen Planung werden mögliche Alternativen, die im Zuge der Implementierung zum Einsatz kommen könnten, für den Fall diskutiert, dass die Implementierung des geplanten scheitert.
In Kapitel 5, der Implementation, wird das geplante Konzept in eine Monitoring-Lösung unter Zuhilfenahme der, im Grundlagen-Kapitel erklärten, Methoden und Konzepte umgesetzt. Die Beschreibung der Implementierung wird dabei unterteilt in Informationsbeschaffung bzw. Erklärung der logischen Abhängigkeiten zwischen den Informationen und die Umsetzung in der gewählten Technologie.
Die Verifikation in Kapitel 6 geht auf die eingangs beschriebenen Ziele der Arbeit ein und prüft die durchgeführte Implementierung auf die Erfüllung der in der Analyse definierten Anforderungen. Nach der Prüfung der Ergebnisse werden die bei der Entwicklung und Implementierung gewonnen Erkenntnisse zusammengefasst und daraus Schlüsse für zukünftige Arbeiten in diesem Themenfeld gezogen. 
Die abschließende Schlussbetrachtung in Kapitel 7 fasst die Arbeit und ihre Ergebnisse auf die eingangs definierten Ziele Bezug nehmend zusammen und gibt einen Ausblick über Möglichkeiten zur Erweiterung der Lösung und dessen Bedienung.