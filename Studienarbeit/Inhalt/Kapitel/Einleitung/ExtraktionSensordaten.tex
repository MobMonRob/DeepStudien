\section{Extraktion der Sensordaten}
Der eingesetzte Sensor liefert
\subsection{Accelerometer}
\subsection{Gyroscop}
Ein Gyroskop (auch Drehraten-Sensor genannt) misst die Rotationsgeschwindigkeit eines Körpers durch drei Freiheitsgrade. Die Rotation um die Längsachse wird als Rollen bezeichnet, die Rotation um die Querachse als Nicken und die Rotation um die Vertikalachse wird als gieren bezeichnet. 
\\
Drehratensensoren können dabei durch Ausnutzen zweier Messprinzipien mechanisch realisert werden. Zum einen kann die Corioliskraft genutzt werden, die näherungsweise proportional zur Drehgeschwindigkeit ist. Die Messgenauigkeit ist dabei für kurze Zeit ausreichend genau, für längere Messungen wird der Drift der Messwerte mit mehreren Grad pro Stunde sehr groß. 
Eine weitaus präzisere, wenngleich aufwändigere Messung ist durch Beobachten des Sagnac Effekts möglich. Hierfür wird ein Lichtstrahl durch einen halbdurchlässigen Spiegel in zwei Teilstrahlen aufgeteilt, die entgegengesetzt durch einen Kreis geführt werden und am Zusammentreffen und ein Interferenzmuster erzeugen. Bei vorhander Drehung muss einer der Lichtstrahlen einen längeren Weg zurücklegen, wodurch das Interferenzmuster anders ist. Durch den Einsatz von Licht liegt der Drift dieses Systems bei weniger als einem Grad pro Stunde. 
\\
Innerhalb einer inertialen Messeinheit für billige, kleine Sensoren ist ene hohe Messgenauigkeit nicht so wichtig, wesewegen dabei nicht der Sagnac Effekt beobachtet wird, sondern in integrierte Schaltkreise realisierte mikro-elektro-mechanische Systeme Andwendung finden. Dabei führt die Verformung einer Feder bei Drehbewegungen zu einer messbaren Kapazitätsveränderung eines Kondensators.
\subsection{Magnetometer}
\subsection{Konvertierung}
Die Sensordaten werden von den Sensoren im Comma-Seperate-Value - Format (CSV) exportiert. Zur Verwendung im JSON-Format entsteht das im folgende beschriebene Programm zur Konvertierung.

%\subsection{Accelerometer}
%\subsection{Gyroscop}\subsection{Gyroscop}
Ein Gyroskop (auch Drehraten-Sensor genannt) misst die Rotationsgeschwindigkeit eines Körpers durch drei Freiheitsgrade. Die Rotation um die Längsachse wird als Rollen bezeichnet, die Rotation um die Querachse als Nicken und die Rotation um die Vertikalachse wird als gieren bezeichnet. 
\\
Drehratensensoren können dabei durch Ausnutzen zweier Messprinzipien mechanisch realisert werden. Zum einen kann die Corioliskraft genutzt werden, die näherungsweise proportional zur Drehgeschwindigkeit ist. Die Messgenauigkeit ist dabei für kurze Zeit ausreichend genau, für längere Messungen wird der Drift der Messwerte mit mehreren Grad pro Stunde sehr groß. 
Eine weitaus präzisere, wenngleich aufwändigere Messung ist durch Beobachten des Sagnac Effekts möglich. Hierfür wird ein Lichtstrahl durch einen halbdurchlässigen Spiegel in zwei Teilstrahlen aufgeteilt, die entgegengesetzt durch einen Kreis geführt werden und am Zusammentreffen und ein Interferenzmuster erzeugen. Bei vorhander Drehung muss einer der Lichtstrahlen einen längeren Weg zurücklegen, wodurch das Interferenzmuster anders ist. Durch den Einsatz von Licht liegt der Drift dieses Systems bei weniger als einem Grad pro Stunde. 
\\
Innerhalb einer inertialen Messeinheit für billige, kleine Sensoren ist ene hohe Messgenauigkeit nicht so wichtig, wesewegen dabei nicht der Sagnac Effekt beobachtet wird, sondern in integrierte Schaltkreise realisierte mikro-elektro-mechanische Systeme Andwendung finden. Dabei führt die Verformung einer Feder bei Drehbewegungen zu einer messbaren Kapazitätsveränderung eines Kondensators.

%\subsection{Magnetometer}
%\input{Inhalt/Kapitel/Einleitung/ExtraktionSensordaten/Konvertierung}