\section{Vorverarbeitung vor der Analyse}
Um die Anzahl der zu analysierenden Dimensionen zu reduzieren, können aus den rohen Sensordaten Informationen wie Rotation oder Beschleunigung berechnet werden. Dies hat den Vorteil, dass die Konstruktion eines Analyse-Systems anhand dieser Informationen gegebenenfalls besser funktioniert, da die Daten vorab visuell durch den Menschen analysiert werden können. Die Vorverarbeitung bringt jedoch auch Nachteile mit sich:

\begin{itemize} 
\item Die Vorverarbeitung nimmt zusätzliche Zeit während der Vorverarbeitung in Anspruch. Statt die Daten direkt zum Analyse-System zu übertragen, müssen gegebenenfalls komplexe Berechnungen durchgeführt werden 
\item Etwaige Ausschläge des Sensors in eine oder mehrere Richtungen beeinflussen das Ergebnis gegebenenfalls stark
\end{itemize}

Folgt man diesem Ansatz werden die folgenden Dimensionen analysiert.
\begin{itemize} 
\item Position (x)
\item Position (y)
\item Position (z)
\item Rotation (x)
\item Rotation (y)
\item Rotation (z)
\end{itemize}

\section{Erkennung auf Basis von Rohdaten}
Die Sensordaten können jedoch auch ohne Vorverarbeitung analysiert werden. Dies verkürzt den gesamten Prozess von der Extraktion bis zur Analyse, da die Daten nach der Extraktion direkt an das Analyse-System übertragen werden können.

\begin{itemize} 
\item Je nach Qualität des Sensors enthalten die extrahierten Sensordaten einzelne Ausschläge. Diese könnten als Anomalien identifiziert werden, was das Ergebnis der Analyse verfälscht
\end{itemize}
Die, im Kapitel X aufgeführten Sensoren Gyroskop, Magnetometer und Accelerometer liefern die folgenden Dimensionen zur Analyse:
\begin{itemize} 
\item Rotationsgeschwindigkeit (x-Richtung)
\item Rotationsgeschwindigkeit (y-Richtung)
\item Rotationsgeschwindigkeit (z-Richtung)
\item Lineare Beschleunigung (x-Richtung)
\item Lineare Beschleunigung (y-Richtung)
\item Lineare Beschleunigung (z-Richtung)
\item Magentische Flussdichte (x-Richtung)
\item Magentische Flussdichte (y-Richtung)
\item Magentische Flussdichte (z-Richtung)
\end{itemize}