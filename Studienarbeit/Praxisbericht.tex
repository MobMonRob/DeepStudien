% ---- Präambel mit Angaben zum Dokument
\documentclass[
fontsize=11pt,
paper=A4,
twoside=false,
listof=totoc,            % Tabellen- und Abbildungsverzeichnis ins Inhaltsverzeichnis
bibliography=totoc,      % Literaturverzeichnis ins Inhaltsverzeichnis aufnehmen
titlepage,               % Titlepage-Umgebung anstatt \maketitle
headsepline,             % horizontale Linie unter Kolumnentitel
abstracton,              % Überschrift einschalten, Abstract muss in {abstract}-Umgebung stehen
]{scrreprt}                  % Verwendung von KOMA-Report
\usepackage[utf8]{inputenc}  % UTF8 Encoding einschalten
\usepackage[ngerman]{babel}  % Neue deutsche Rechtschreibung
\usepackage[T1]{fontenc}     % Ausgabe von westeuropäischen Zeichen (auch Umlaute)
\usepackage{graphicx}        % Einbinden von Grafiken erlauben
\usepackage{tikz}
\usetikzlibrary{er,positioning}
\usepackage{verbatim} 
\usepackage{xcolor}
\usepackage{setspace}        % Zeilenabstand \singlespacing, \onehalfspaceing, \doublespacing
\usepackage{listings}
\usepackage[
%showframe,              % Ränder anzeigen lassen
left=2.5cm, right=2.5cm,
top=2.5cm,  bottom=2.5cm,
includeheadfoot
]{geometry}                      % Seitenlayout einstellen
\usepackage{mathpazo}            % Einstellung der verwendeten Schriftarten
\usepackage{scrpage2}            % Gestaltung von Fuß- und Kopfzeilen
\usepackage[hidelinks]{hyperref} % Klickbare Links im Dokument
\urlstyle{same}                  % Aktuelle Schrift auch für URLs
\usepackage{acronym}             % Abkürzungen, Abkürzungsverzeichnis
\usepackage{titletoc}            % Anpassungen am Inhaltsverzeichnis
\usepackage{tikz}
\usetikzlibrary{shapes.geometric, arrows}
\contentsmargin{0.7cm}           % Abstand im Inhaltsverzeichnis zw. Punkt und Seitenzahl


\include{htmldef}

% start of JSON 

\colorlet{punct}{red!60!black}
\definecolor{background}{HTML}{EEEEEE}
\definecolor{delim}{RGB}{20,105,176}
\colorlet{numb}{magenta!60!black}

\lstdefinelanguage{json}{
	basicstyle=\normalfont\ttfamily,
	numbers=left,
	numberstyle=\scriptsize,
	numbersep=8pt,
	showstringspaces=false,
	breaklines=true,
	frame=lines,
	literate=
	*{0}{{{\color{numb}0}}}{1}
	{1}{{{\color{numb}1}}}{1}
	{2}{{{\color{numb}2}}}{1}
	{3}{{{\color{numb}3}}}{1}
	{4}{{{\color{numb}4}}}{1}
	{5}{{{\color{numb}5}}}{1}
	{6}{{{\color{numb}6}}}{1}
	{7}{{{\color{numb}7}}}{1}
	{8}{{{\color{numb}8}}}{1}
	{9}{{{\color{numb}9}}}{1}
	{:}{{{\color{punct}{:}}}}{1}
	{,}{{{\color{punct}{,}}}}{1}
	{\{}{{{\color{delim}{\{}}}}{1}
	{\}}{{{\color{delim}{\}}}}}{1}
	{[}{{{\color{delim}{[}}}}{1}
	{]}{{{\color{delim}{]}}}}{1},
}

% end of JSON

% ---- Für Anführungszeichen und Zitate
\usepackage[babel,german=quotes]{csquotes} % Deutsche Anführungszeichen + Zitate

% ---- Für Mathevorlage
\usepackage{amsmath}    % Erweiterung vom Mathe-Satz
\usepackage{amssymb}    % Lädt amsfonts und weitere Symbole
\usepackage{MnSymbol}   % Für Symbole, die in amssymb nicht enthalten sind.

% ---- Für Quellcodevorlage
\usepackage{scrhack}    % Hack to use listings in KOMA-Script
\usepackage{listings}   % Datstellung von Quellcode
\usepackage{xcolor}     % einfache Verwendung von Farben

\usepackage{amsmath}
\usepackage{algorithm}
\usepackage[noend]{algpseudocode}

\usepackage{cite}


% ------ Eigene Farben für den Quellcode
\definecolor{EigenesLila}{rgb}{0.4,0.1,0.4}
\definecolor{EigenesCyan}{rgb}{0.0,0.5,0.4}
\definecolor{EigenesGruen}{rgb}{0.3,0.5,0.4}
\definecolor{EigenesBlau}{rgb}{0.0,0.0,1.0}

% ------ Default Listing-Styles

% Breche lange Zeilen um + Symbol für Zeilenumbruch einfügen
\lstset{ 
	breaklines=true,
	breakatwhitespace=true,
	prebreak=\raisebox{0ex}[0ex][0ex]{\ensuremath{\rhookswarrow}},
	postbreak=\raisebox{0ex}[0ex][0ex]{\ensuremath{\rcurvearrowse\space}}
}
\lstset{
	tabsize=4 % Setze die Breite eines Tabs
}

% ------ Eigener JAVA-Style für den Quellcode
\renewcommand{\ttdefault}{pcr}                 % Schriftart, welche auch fett beinhaltet
\lstdefinestyle{EigenerJavaStyle}{
	language=Java,                             % Syntax Highlighting für Java
	columns=flexible,                          % Besseres Schriftbild
	numbers=left,                              % Nummerierung der Zeilen
	%frame=single,                             % Umrandung des Codes
	showstringspaces=false,                    % Keine Leerzeichen hervorheben
	basicstyle=\ttfamily,                      % Grundsätzlicher Schriftstyle
	keywordstyle=\bfseries\color{EigenesLila}, % Keywords in eigener Farbe
	commentstyle=\itshape\color{EigenesGruen}, % Kommentare in eigener Farbe
	stringstyle=\color{EigenesBlau},           % Strings in eigener Farbe
	xleftmargin=5.0ex
}


% ---- Nicht benötigte Packages
\usepackage{lipsum}      % Dummy Text Lorem ipsum
%\usepackage{ucs}        % Erweiterte UTF8 Encoding Unterstützung
%\usepackage{textcomp}   % Einsatz von Eurozeichen u. a. Symbolen
%\usepackage{adjustbox}  % Allign Pictures(left, right, center)
%\usepackage{pifont}     % http://ctan.org/pkg/pifont
%\usepackage{url}        % Zum Verlinken (\url{...} und \href{...}{...})
%\usepackage{nameref}    % Zum Referenzieren von Überschriften als Text

% ---- Hilfreiches
\newcommand{\zB}{z.\,B. }
\newcommand{\dash}{d.\,h. }

% ---- Elektronische Version oder Gedruckte Version?
\usepackage{ifthen}
\newboolean{e-Abgabe}
\setboolean{e-Abgabe}{false}    % false=gedruckte Fassung

% ---- Abstand verkleinern von der Überschrift 
\renewcommand*{\chapterheadstartvskip}{\vspace*{.5\baselineskip}}

% ---- Persönlichen Daten:
\newcommand{\titel}{Ansätze zur Erkennung von Anomalien in Sensordaten aus mobilen Plattformen auf Basis von Deep Learning-Algorithmen}
\newcommand{\titelheader}{Erkennung von Sensoranomalien mit Deep Learning}
\newcommand{\arbeit}{Studienarbeit}
\newcommand{\studiengang}{Angewandte Informatik}
\newcommand{\autor}{Benedikt Bosshammer}
\newcommand{\autorWith}{Enrico Kaack}
\newcommand{\autorReverse}{Bosshammer, Benedikt}
\newcommand{\verfassungsort}{Karlsruhe}
\newcommand{\matrikelnr}{2416344}
\newcommand{\kurs}{TINF15B2}
\newcommand{\bearbeitungsmonat}{September 2017}
\newcommand{\abgabe}{15. Mai 2018}
\newcommand{\bearbeitungszeitraum} {01.10.2017 - 23.12.2017}
\newcommand{\bearbeitungszeitraumNd} {15.02.2018 - 15.05.2018}
%\newcommand{\firmaname}{SAP Deutschland}
%\newcommand{\firmastrasse}{Tesdorpfstraße 6}
%\newcommand{\firmaplz}{69190 Hamburg, Deutschland}
%\newcommand{\abteilung}{SAP Consulting - Analytics\&Insights}
\newcommand{\betreuer}{Prof. Dr. Marcus Strand}

% ---- Metainformation für das PDF Dokument
\hypersetup{
	pdftitle    = {\titel},
	pdfsubject  = {\arbeit},
	pdfauthor   = {\autor},
	%pdfkeywords = {Keywords angeben},
	pdfcreator  = {LaTeX},
	%pdfproducer = {in der Regel pdfTeX}
}

% ---- Definition der Kopf- und Fußzeilen
\clearscrheadfoot                               % Löschen von LaTeX Standard
\automark[section]{chapter}                     % Füllen von section und chapter
\renewcommand*{\chaptermarkformat}{}            % Entfernt die Kapitelnummer
\renewcommand*{\sectionmarkformat}{}            % Entfernt die Sectionnummer
% Angaben [für "plain"]{für "scrheadings"}
\ihead[]{\titelheader}                          % Kopfzeile links
\chead[]{}                                      % Kopfzeile mitte
\ohead[]{\rightmark}                            % Kopfzeile rechts
\ifoot[]{}                                      % Fußzeile links
\cfoot[\sffamily\pagemark]{\sffamily\pagemark}  % Fußzeile mitte
\ofoot[]{}                                      % Fußzeile rechts
\setheadsepline{0.2pt}                          % Liniendicke Kopfzeile
\setfootsepline{0.0pt}                          % Liniendicke Fußzeile

% ---- Beginn des Dokuments
\begin{document}
	\setlength{\parindent}{0pt}
	\setcounter{secnumdepth}{3}    % Nummerierungstiefe fürs Inhaltsverzeichnis
	\setcounter{tocdepth}{3}       % Tiefe des Inhaltsverzeichnisses
	\sffamily                      % Serifenlose Schrift verwenden.
	% Für serife Schrift auskommentieren.
	
	% ---- Titelseite
	\singlespacing
	\thispagestyle{empty}
\begin{titlepage}
\enlargethispage{4cm}

\begin{center}                  
	\includegraphics[height=2.5cm]{Bilder/Logos/Logo_DHBW.pdf} 
\end{center} 
\vspace*{0.1cm}

\begin{center}
	\huge{\textbf{\titel}}\\[1.5cm]
	\Large{\textbf{\arbeit}}\\[0.5cm]
	\normalsize{im Rahmen der Prüfung zum\\[1ex] \textbf{Bachelor of Science (B.Sc.)}}\\[0.5cm]
	\Large{des Studienganges \studiengang}\\[1ex]
	\normalsize{an der Dualen Hochschule Baden-Württemberg Karlsruhe}\\[1cm]
	\normalsize{von}\\[1ex] \Large{\textbf{\autor}} \\[0.5cm] \normalsize{zusammen mit}\\[1ex] \Large{\textbf{\autorWith}} \\[1cm]
%	\normalsize{\bearbeitungsmonat}\\[1ex] \Large{\textbf{-Sperrvermerk-}}\\[0.5cm]
\end{center}

\begin{center}
	\vfill
	\begin{tabular}{ll}
		Abgabedatum:                    & \abgabe \\[0.2cm]
		Bearbeitungszeitraum:           & \bearbeitungszeitraum \\[0.2cm]
							            & \bearbeitungszeitraumNd \\[0.2cm]
		Matrikelnummer, Kurs:           & \matrikelnr, \kurs \\ [0.2cm]
		Wissenschaftlicher Betreuer:    & \betreuer \\[1.5cm]
	\end{tabular} 
\end{center}
\end{titlepage} % Titelseite
	\newcounter{savepage}
	\pagenumbering{Roman}                % Römische Seitenzahlen
	\onehalfspacing
	
	% ---- Erklärung, Sperrvermerk, Abstact
	\chapter*{Eidesstattliche Erklärung}
Ich versichere hiermit, dass ich meine Projektarbeit mit dem Thema:
\begin{quote}
	\textit{\titel}
\end{quote} 
gemäß § 5 der \enquote{Studien- und Prüfungsordnung DHBW Technik} vom 29. September 2015 selbstständig verfasst und keine anderen als die angegebenen Quellen und Hilfsmittel benutzt habe. Ich versichere zudem, dass die eingereichte elektronische Fassung mit der gedruckten Fassung übereinstimmt.

\vspace{1cm}

\verfassungsort, den \today \\[0.5cm]
\ifthenelse{\boolean{e-Abgabe}}
	{\underline{Gez. \autor}}
	{\makebox[6cm]{\hrulefill}}\\ 
\autorReverse

	%\chapter*{Sperrvermerk}
Die nachfolgende Arbeit enthält vertrauliche Daten der:
\begin{quote}
	\firmaname

	\firmastrasse

	\firmaplz

\end{quote}

\vspace{0.5cm}

Sie darf als Leistungsnachweis des Studienganges \studiengang{} 2015 an der DHBW Karlsruhe verwendet und nur zur Prüfungszwecken zugänglich gemacht werden. Über den Inhalt ist Stillschweigen zu bewahren. Veröffentlichungen oder Vervielfältigungen der Projektarbeit - auch auszugsweise - sind ohne ausdrückliche Genehmigung der SAP SE nicht gestattet.

\vspace{0.5cm}

SAP und die SAP Logos sind eingetragene Warenzeichen der SAP SE. Die Wiedergabe von Gebrauchsnamen, Handelsnamen, Warenbezeichnungen usw. in dieser Arbeit berechtigt auch ohne besondere Kennzeichnung nicht zu der Annahme, dass solche Namen im Sinne der Warenzeichen-und Markenschutz-Gesetzgebung als frei zu betrachten wären und daher von jedem benutzt werden dürfen.
	\setcounter{savepage}{\value{page}}
	
	% ---- Inhaltsverzeichnis
	\renewcommand*{\chapterpagestyle}{empty}
	\pagestyle{empty}
	\singlespacing
	\tableofcontents
	
	% ---- Inhalt der Studienarbeit
	\cleardoublepage
	\pagenumbering{arabic}   % Arabische Seitenzahlen für den Hauptteil
	\rmfamily
	\renewcommand*{\chapterpagestyle}{scrheadings}
	\pagestyle{scrheadings}
	\onehalfspacing
	
	
	\chapter{Einleitung}
\section{Motivation}
Diese Arbeit entsteht als Teil des Forschungsprojekts NAMEDESPROJEKTS. Neben der Voraussage von Eigenschaften wie Verschleiß entstehen auch Szenarien in denen die Erkennung von unübliche Vorkommnisse in bekannten und routinierten Abläufen das Resultat von Predictive Maintenance positiv beeinflussen kann. Ziel dieser Arbeit ist die Entwicklung von Ansätzen um jene Vorkommnisse innerhalb eines bekannten Ablaufs zu identifizieren.
\section{Szenario}
Um den abstrakten Begriff Vorkommnis in eine konkrete Problemstellung zu überführen, entsteht als Basis der nachfolgenden Ansätze das in Abbildung X dargestellte Szenario. \newline
Ein mobiles System mit drei Achsen soll entlang einer Strecke ein Hindernisse wie z.B. Kabel überfahren. Während die Fahrt ohne Hindernis im Kontext der Anomalie-Erkennung der bekannten bzw. routinierten Strecke entspricht, stellt das Überfahren des Hindernisses die Anomalie im dargestellten Szenario dar. 
\section{Geplantes Vorgehen}
Im ersten Schritt sollen die verfügbaren Sensoren identifiziert und ein Verfahren zur unkomplizierten Extraktion der Daten, die jene liefern, entwickelt werden. Da diese Sensordaten die Basis für den Entwurf, die Implementierung und die Bewertung der Ansätze zur Anomalie-Erkennung bilden, lohnt sich die Investition in die Planung und Entwicklung eines geeigneten Verfahren zur Extraktion. Teil der Extraktion soll dabei - falls notwendig - auch die Konvertierung in das Format JSON sein, da sich diese innerhalb von Java und Python übersichtlicher und mit weniger Aufwand verarbeiten lassen. \newline
Steht eine extrahierte und konvertierte Datenbasis zur Verfügung folgt der Entwurf und anschließende Bewertung möglicher Ansätze. Wurden ein oder gegebenenfalls mehrere Ansätze identifiziert, folgt die Entscheidung für einer Technologie und der Entwurf für die Umsetzung. Im letzten Schritt wird eine Lösung auf Basis des Entwurfs entwickelt und mit existierenden Ansätzen verglichen.

\begin{algorithm}
\caption{Ablauf Konvertierung}\label{euclid}
\begin{algorithmic}[1]
\State $\textit{input} \gets \text{lese }\textit{input }\text{file}$
\State $\textit{output} \gets \text{erstelle } \textit{output } \text{file}$
\If {$i > \textit{stringlen}$} \Return false
\EndIf
\State $j \gets \textit{patlen}$
\BState \emph{loop}:
\If {$\textit{string}(i) = \textit{path}(j)$}
\State $j \gets j-1$.
\State $i \gets i-1$.
\State \textbf{goto} \emph{loop}.
\State \textbf{close};
\EndIf
\State $i \gets i+\max(\textit{delta}_1(\textit{string}(i)),\textit{delta}_2(j))$.
\State \textbf{goto} \emph{top}.
\end{algorithmic}
\end{algorithm}
\input{Inhalt/Kapitel/Einleitung/ExtraktionSensordaten}

	\section{Vorverarbeitung vor der Analyse}
Um die Anzahl der zu analysierenden Dimensionen zu reduzieren, können aus den rohen Sensordaten Informationen wie Rotation oder Beschleunigung berechnet werden. Dies hat den Vorteil, dass die Konstruktion eines Analyse-Systems anhand dieser Informationen gegebenenfalls besser funktioniert, da die Daten vorab visuell durch den Menschen analysiert werden können. Die Vorverarbeitung bringt jedoch auch Nachteile mit sich:

\begin{itemize} 
\item Die Vorverarbeitung nimmt zusätzliche Zeit während der Vorverarbeitung in Anspruch. Statt die Daten direkt zum Analyse-System zu übertragen, müssen gegebenenfalls komplexe Berechnungen durchgeführt werden 
\item Etwaige Ausschläge des Sensors in eine oder mehrere Richtungen beeinflussen das Ergebnis gegebenenfalls stark
\end{itemize}

Folgt man diesem Ansatz werden die folgenden Dimensionen analysiert.
\begin{itemize} 
\item Position (x)
\item Position (y)
\item Position (z)
\item Rotation (x)
\item Rotation (y)
\item Rotation (z)
\end{itemize}

\section{Erkennung auf Basis von Rohdaten}
Die Sensordaten können jedoch auch ohne Vorverarbeitung analysiert werden. Dies verkürzt den gesamten Prozess von der Extraktion bis zur Analyse, da die Daten nach der Extraktion direkt an das Analyse-System übertragen werden können.

\begin{itemize} 
\item Je nach Qualität des Sensors enthalten die extrahierten Sensordaten einzelne Ausschläge. Diese könnten als Anomalien identifiziert werden, was das Ergebnis der Analyse verfälscht
\end{itemize}
Die, im Kapitel X aufgeführten Sensoren Gyroskop, Magnetometer und Accelerometer liefern die folgenden Dimensionen zur Analyse:
\begin{itemize} 
\item Rotationsgeschwindigkeit (x-Richtung)
\item Rotationsgeschwindigkeit (y-Richtung)
\item Rotationsgeschwindigkeit (z-Richtung)
\item Lineare Beschleunigung (x-Richtung)
\item Lineare Beschleunigung (y-Richtung)
\item Lineare Beschleunigung (z-Richtung)
\item Magentische Flussdichte (x-Richtung)
\item Magentische Flussdichte (y-Richtung)
\item Magentische Flussdichte (z-Richtung)
\end{itemize}
	\chapter{Vorverarbeitung}
\section{Grundlagen}
\subsection{Deep Learning}
\subsection{Artifical Neural Networks}
\subsection{Convolutional Neural Networks}
\subsection{Recurrent Neural Networks}
\subsection{Auto Encoder}
\subsection{Abwägung}
	\chapter{Erkennung}
\section{Grundlagen}
\subsection{Deep Learning}
\subsection{Artifical Neural Networks}
\subsection{Convolutional Neural Networks}
\subsection{Recurrent Neural Networks}
\subsection{Auto Encoder}
\subsection{Abwägung}
	\include{Inhalt/Kapitel/Ergebnis}
	\include{Inhalt/Kapitel/Fazit}
	
	\cleardoublepage
	\renewcommand*{\chapterpagestyle}{plain}
	\pagestyle{plain}
	\pagenumbering{Roman}                   % Römische Seitenzahlen
	\setcounter{page}{\value{savepage}}
	
	% ---- Verzeichnisse
	\listoffigures                          % Erzeugen des Abbildungsverzeichnisses 
	%\listoftables                           % Erzeugen des Tabellenverzeichnisses
	\renewcommand{\lstlistlistingname}{Listenverzeichnis}
	\lstlistoflistings                      % Erzeugen des Listenverzeichnisses
	\chapter*{Abkürzungsverzeichnis}
\addcontentsline{toc}{chapter}{Abkürzungsverzeichnis} 

\begin{acronym}[HTML] % längstes Kürzel wird verwendet für den Abstand (Kürzel zu Beschreibung)

	% Alphabetisch sortieren!
	\acro{API}{Application Programming Interface}
	\acro{URI}{Uniform Resource Identifier}
	\acro{SPA}{Single Page Application}
	\acro{SCP}{SAP Cloud Platform}
	\acro{XML}{Extended Markup Language}
	\acro{HTML}{Hypertext Markup Language}
\end{acronym}
	
	% ---- Literaturverzeichnis
	 \nocite{*}
	 \bibliographystyle{ieeetr}
	 \bibliography{Inhalt/literatur}         % Bib-Datei ohne .bib Endung
	
	
	% ---- Anhang
	\appendix
	%\clearpage
	%\pagenumbering{Roman}  % römische Seitenzahlen für Anhang
	
	\newpage
\end{document}
